%!TEX encoding = IsoLatin
\documentclass[conference]{IEEEtran}

\usepackage{cite}
\usepackage{graphicx}
\usepackage{amsmath}
\usepackage{algorithmic}
\usepackage{fixltx2e}
\usepackage{stfloats}
\usepackage{url}
\usepackage{hyperref}
\usepackage[latin1]{inputenc}
\usepackage{lipsum}
\usepackage{coolMath}
\title{Relevance Vector Machine for regression and comparaison with the SVR}
\author{
	\IEEEauthorblockN{Hadrien Hendrikx}
	\IEEEauthorblockA{
					School of Engineering (STI)\\
					\'Ecole Polytechnique \\ F�d�rale de Lausanne\\
					Lausanne, Switzerland\\
					hadrien.hendrix@epfl.ch
					}\\
	\and
	\IEEEauthorblockN{Gregoire Gallois-Montbrun}
	\IEEEauthorblockA{
					School of Engineering (STI)\\
					\'Ecole Polytechnique \\ F�d�rale de Lausanne\\
					Lausanne, Switzerland\\
					gregoire.gallois-montbrun@epfl.ch
					}
	\and
	\IEEEauthorblockN{Louis Faury}
	\IEEEauthorblockA{
					School of Engineering (STI)\\
					\'Ecole Polytechnique \\ F�d�rale de Lausanne\\
					Lausanne, Switzerland\\
					louis.faury@epfl.ch
					}
	}
	
\begin{document}

	\maketitle
	
	\begin{abstract}
		\lipsum[1]
	\end{abstract}
	
	\section{Introduction}
	{
	
	}
	\section{Theoretical background}
	{
		\subsection{Support Vector machine for Regression}
		{
			The Support Vector machine for Regression (SVR) extends the SVM method for regression tasks. Let us first consider the simple linear regression case for a dataset $\{X,\mathbf{t}\} = \{ (x_1,\hdots,x_N)^T, (t_1,\hdots,t_n)^T\}$, where we minimize a regularized error function given by : 
			\begin{equation}
				E(w) = \frac{1}{2}\sum_{n=1}^N\left\{w^T\phi(x_n) - t_n\right\}^2 + \frac{\lambda}{2}\lVert w \rVert ^2
			\end{equation}
			To obtain sparse solution, the quadratic term is replaced by a $\eps$-\emph{insensitive error function} (see \cite{vapnik1995nature}) denoted $E_\eps(\cdot)$ with : 
			\begin{equation}
				\label{eq::loss}
				E_\eps(y(x)-t) = \left\{
								\begin{aligned}
								&0, \quad &\text{ if } \vert y(x)-t\vert < \eps\\
								& \vert y(x)-t\vert < \eps \quad &\text{otherwise}  
								\end{aligned}\right.
			\end{equation}
			Therefore, the quantity to be minimized can be expressed as : 
			\begin{equation}
				C\sum_{n=1}^N E_\eps(w^T\phi(x_n)-t) + \frac{1}{2}\lVert w \rVert ^2
			\end{equation}
			where $C$ is a regularization parameter. 
			\newline 
			As for the SVM, one can introduce \emph{slack variables} in order to transform this optimization program into a quadratic programming problem (quadratic objective, linear constraints). Once the problem solved, predictions are made using : 
			\begin{equation}
				y(x) = \sum_{n=1}^N (a_n - \hat{a}_n)k(x,x_n) + b
			\end{equation}
			where we introduced the kernel $k(x,x') = \phi(x)^T\phi(x')$. The coefficients $\{a_n\}$ and $\{\hat{a}_n\}$ actually are Lagrange multipliers for the QP problem, and provide a \emph{sparse} solutions in the data-points. The \emph{support vectors} (data point used for predictions) are those for which $a_n \neq 0$ or $\hat{a}_n\neq 0$, in other words those that lie on the boundary or outside of the $\eps$-tube defined by the loss function in equation (\ref{eq::loss}). \newline
			As for the SVM, one can adopt a $\nu$-SVR formulation to have an lower-bound control on the number of retained support vectors. 
		}
		\subsection{Relevance Vector machine for Regression}
		{
			The SVR therefore provides a useful tool for obtaining sparse regression machine. However, it suffers from a \emph{number of limitations}, such as an output representation as decision rather than posterior probability, the need to estimate hyper-parameters (kernel width, penalization parameter) via \emph{held-out methods} (like cross-validation), or the need for the kernel to be a Mercer kernel type (positive definite). 
			\newline
			The Relevance Vector Machine for regression is a \emph{Bayesian sparse kernel technique} that shares many of the SVR's characteristics while avoiding its limitations. It instantiates a model intended to mirror the structure of the SVR : 
			\begin{equation}
				y(x) = \sum_{n=1}^{N+1}w_nk(x,x_n)
			\end{equation}
			where the bias $b$ is included in the predictor $w$ and $k(\cdot,\cdot)$ is an arbitrary kernel (not necessarily positive definite). Assuming i.i.d data sample with Gaussian noise of precision $\beta$, the likelihood writes : 
			\begin{equation}
				\condp{\mathbf{t}}{X,w,\beta} = \prod_{n=1}^N \normalDb{t_n}{y(x_n)}{\beta}
			\end{equation}
			The predictor $w$ is given a centered Gaussian prior distribution :
			\begin{equation}
				\condp{w}{\alpha} = \prod_{i=1}^{N+1} \normalDb{w}{0}{\alpha_i^{-1}}
			\end{equation}
			introducing a separate precision parameter $\alpha_i$ for each weight parameter $w_i$. 
			\newline 
			This leads to a Gaussian posterior distribution over $w$ : 
			\begin{equation}
				\begin{aligned}
					\condp{w}{\mathbf{t},X,\alpha,\beta} &= \condp{\mathbf{t}}{w,X,\beta}\condp{w}{\alpha}\\
												&= \normalDb{w}{m}{\Sigma}
				\end{aligned}
			\end{equation}
			where 
			\begin{equation}
				\begin{aligned}
					m &= \beta \Sigma \boldsymbol{\phi}^T t\\
					\Sigma &= \left(A+\beta \boldsymbol{\phi}^T\boldsymbol{\phi}\right)^{-1}, \qquad A = \text{diag}(\alpha_i)
				\end{aligned}
			\end{equation}
			In a full Bayesian approach, $\alpha$ and $\beta$ are both given prior distributions. However, this leads to intractable computations when computing predictions. The use of \emph{empirical Bayes} solves this problem, by approximating $\alpha$ and $\beta$ by their maximum-a-posteriori value (also known as the \emph{evidence approximation}). \newline
			As a result of approximation, a proportion of parameters $\alpha_i$ are driven to infinite values, constraining the corresponding weights $w_i$ to have $0$ mean and infinite precision, and hence are set to $0$. The resulting predictions are therefore sparse in data-points, and the inputs $\{x_n\}$ corresponding to non-zero weights are called \emph{relevance vectors}. Once the optimal values $\alpha^*$ and $\beta^*$ found, the predictive distribution over $y$ can therefore be computed using $\alpha^*$ and $\beta^*$. 
			\newline The sparsity analysis of the RVR leads to a practical algorithm for optimizing the hyper-parameters that has significant speed advantages, and is referred to as \emph{automatic relevance determination}. The full algorithm and its justification can be found in \cite{bishop2006pattern}.
		}
		\subsection{Theoretical method comparaison}
		{
			\subsubsection{Complexity}
			{
				When comparing the SVR's and the RVR's complexity, one must distinguish complexity at training time and at testing time. \newline
				Training the SVR sums up in solving a large quadratic-programing (QP) problem. A popular approach to do that implies breaking up the initial QP in smaller problems, solvable analytically, and is called \emph{Sequential Minimization Optimization} (SMO). It requires a linear (with respect to the datapoints) amount of memory, and scales between linear and quadratic complexity in the training set size (see \cite{platt1998sequential}). At test time, the complexity is linear in the number of support vectors. \newline
				Training a RVM involves optimizing a non-convex function. For a model with $M$ basis functions, the RVM requires the inversion of a $M\times M$ matrix, which requires from $O(M^{2.7})$ to $O(M^{3})$, which is as we just saw larger than the SVR's cost. However, parameters are determined automatically and in one run when training a RVR, while the hyper-parameters typically need several runs (f-fold cross-validation) to be estimated. At testing time, the RVR's complexity grows linearly with the number of relevance vectors.  
			}
			\subsection{Performance}
			{
				We just named one of the major pros of using RVR - there is no need for using held-out methods to estimate hyper-parameters, as they are automatically determined through automatic relevance detection (expect for parametric basis functions). Also, it has been shown (\cite{platt1998sequential}), \cite{bishop2006pattern}) that RVR leads to \emph{sparser solutions, without loss of generalization abilities} (on the contrary, the RVR usually performs better than the SVR). We will try to observe this observations in the experiments derived hereinafter.
			}
		}
	}
	\section{Results}
	{
		\subsection{Datasets presentation}
		{

		}
		\subsection{Results}
		{

		}
	}
	\section{Discussion}
	{

	}
	\section{Conclusion}
	{

	}
	
	\bibliographystyle{plain}
 	 \bibliography{bib_file}
	
\end{document}