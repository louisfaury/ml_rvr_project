%!TEX encoding = IsoLatin
\documentclass[conference]{IEEEtran}

\usepackage{cite}
\usepackage{graphicx}
\usepackage{amsmath}
\usepackage{algorithmic}
\usepackage{fixltx2e}
\usepackage{stfloats}
\usepackage{url}
\usepackage{hyperref}
\usepackage[latin1]{inputenc}
\usepackage{lipsum}
\usepackage{coolMath}
\title{Relevance Vector Machine for regression and comparaison with the SVR}
\author{
	\IEEEauthorblockN{Hadrien Hendrikx}
	\IEEEauthorblockA{
					School of Engineering (STI)\\
					\'Ecole Polytechnique \\ F�d�rale de Lausanne\\
					Lausanne, Switzerland\\
					hadrien.hendrix@epfl.ch
					}\\
	\and
	\IEEEauthorblockN{Gregoire Gallois-Montbrun}
	\IEEEauthorblockA{
					School of Engineering (STI)\\
					\'Ecole Polytechnique \\ F�d�rale de Lausanne\\
					Lausanne, Switzerland\\
					gregoire.gallois-montbrun@epfl.ch
					}
	\and
	\IEEEauthorblockN{Louis Faury}
	\IEEEauthorblockA{
					School of Engineering (STI)\\
					\'Ecole Polytechnique \\ F�d�rale de Lausanne\\
					Lausanne, Switzerland\\
					louis.faury@epfl.ch
					}
	}
	
\begin{document}

	\maketitle
	
	\begin{abstract}
		\lipsum[1]
	\end{abstract}
	
	\section{Introduction}
	{
	
	}
	\section{Theoretical background}
	{
		\subsection{Support Vector machine for Regression}
		{
			The Support Vector machine for Regression (SVR) extends the SVM method for regression tasks. Let us first consider the simple linear regression case for a dataset $\{X,\mathbf{t}\} = \{ (x_1,\hdots,x_N)^T, (t_1,\hdots,t_n)^T\}$, where we minimize a regularized error function given by : 
			\begin{equation}
				E(w) = \frac{1}{2}\sum_{n=1}^N\left\{w^T\phi(x_n) - t_n\right\}^2 + \frac{\lambda}{2}\lVert w \rVert ^2
			\end{equation}
			To obtain sparse solution, the quadratic term is replaced by a $\eps$-\emph{insensitive error function} (see \cite{vapnik1995nature}) denoted $E_\eps(\cdot)$ with : 
			\begin{equation}
				\label{eq::loss}
				E_\eps(y(x)-t) = \left\{
								\begin{aligned}
								&0, \quad &\text{ if } \vert y(x)-t\vert < \eps\\
								& \vert y(x)-t\vert < \eps \quad &\text{otherwise}  
								\end{aligned}\right.
			\end{equation}
			Therefore, the quantity to be minimized can be expressed as : 
			\begin{equation}
				C\sum_{n=1}^N E_\eps(w^T\phi(x_n)-t) + \frac{1}{2}\lVert w \rVert ^2
			\end{equation}
			where $C$ is a regularization parameter. 
			\newline 
			As for the SVM, one can introduce \emph{slack variables} in order to transform this optimization program into a quadratic programming problem (quadratic objective, linear constraints). Once the problem solved, predictions are made using : 
			\begin{equation}
				y(x) = \sum_{n=1}^N (a_n - \hat{a}_n)k(x,x_n) + b
			\end{equation}
			where we introduced the kernel $k(x,x') = \phi(x)^T\phi(x')$. The coefficients $\{a_n\}$ and $\{\hat{a}_n\}$ actually are Lagrange multipliers for the QP problem, and provide a \emph{sparse} solutions in the data-points. The \emph{support vectors} (data point used for predictions) are those for which $a_n \neq 0$ or $\hat{a}_n\neq 0$, in other words those that lie on the boundary or outside of the $\eps$-tube defined by the loss function in equation (\ref{eq::loss}). \newline
			As for the SVM, onecan adopt a $\nu-SVR$ formulation to have an lower-bound control on the number of retained support vectors. 
		}
		\subsection{Relevance Vector machine for Regression}
		{
		
		}
	}
	\section{Results}
	{
		\subsection{Datasets presentation}
		{

		}
		\subsection{Results}
		{

		}
	}
	\section{Discussion}
	{

	}
	\section{Conclusion}
	{

	}
	
	\bibliographystyle{plain}
 	 \bibliography{bib_file}
	
\end{document}